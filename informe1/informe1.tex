%----------------------------------------------------------------------------------------
%	PACKAGES AND OTHER DOCUMENT CONFIGURATIONS
%----------------------------------------------------------------------------------------

\documentclass[twoside,twocolumn,a4paper]{article}

\usepackage{blindtext} % Package to generate dummy text throughout this template 

\usepackage[T1]{fontenc} % Use 8-bit encoding that has 256 glyphs

\linespread{1.05} % Line spacing - Palatino needs more space between lines
\usepackage{microtype} % Slightly tweak font spacing for aesthetics

\usepackage[spanish]{babel} % Language hyphenation and typographical rules

\usepackage[numbib,notlof,notlot,nottoc]{tocbibind} % Shows bibliography as a section

\usepackage[hmarginratio=1:1,top=32mm,columnsep=20pt]{geometry} % Document margins

\usepackage[hang, small,labelfont=bf,up,textfont=up]{caption} % Custom captions under/above floats in tables or figures

\usepackage{booktabs} % Horizontal rules in tables

\usepackage{enumitem} % Customized lists

\setlist[itemize]{noitemsep} % Make itemize lists more compact

\usepackage{abstract} % Allows abstract customization

\renewcommand{\abstractnamefont}{\normalfont\bfseries} % Set the "Abstract" text to bold

\usepackage{fancyhdr} % Headers and footers
\pagestyle{fancy} % All pages have headers and footers
\fancyhead{} % Blank out the default header
\fancyfoot{} % Blank out the default footer
\fancyhead[C]{Laboratorio 3 $\bullet$ Informe 1 $\bullet$ Inafuku, Petino, Poggi} % Custom header text
\fancyfoot[C]{\thepage} % Custom footer text

\usepackage{titling} % Customizing the title section

\usepackage{hyperref} % For hyperlinks in the PDF

%----------------------------------------------------------------------------------------
%	TITLE SECTION
%----------------------------------------------------------------------------------------

\setlength{\droptitle}{-4\baselineskip} % Move the title up

\pretitle{\begin{center}\LARGE\bfseries} % Article title formatting
\posttitle{\end{center}} % Article title closing formatting
\title{Medici\'on de la resistencia de una l\'ampara incandescente} % Article title
\author{%
\textsc{Maximiliano Inafuku} \\[1ex] % Your name
\normalsize \href{mailto:maxi-46@hotmail.com}{maxi-46@hotmail.com} % Your email address
\and % Uncomment if 2 authors are required, duplicate these 4 lines if more
\textsc{Ernesto Petino} \\[1ex] % Second author's name
\normalsize \href{mailto:ernesto.atmo@gmail.com}{ernesto.atmo@gmail.com} % Second author's email address
\and % Uncomment if 2 authors are required, duplicate these 4 lines if more
\textsc{Ignacio Poggi} \\[1ex] % Second author's name
\normalsize \href{mailto:ignaciop.3@gmail.com}{ignaciop.3@gmail.com} % Second author's email address
}

\date{Laboratorio 3, C\'atedra Bilbao - Departamento de F\'isica, Facultad de Ciencias Exactas y Naturales, Universidad de Buenos Aires \newline \\ \today} % Leave empty to omit a date
\renewcommand{\maketitlehookd}{%
\begin{abstract}
\noindent \blindtext % Dummy abstract text - replace \blindtext with your abstract text
\end{abstract}
}

%----------------------------------------------------------------------------------------

\begin{document}

% Print the title
\maketitle

%----------------------------------------------------------------------------------------
%	ARTICLE CONTENTS
%----------------------------------------------------------------------------------------

\section{Introducci\'on}


\blindtext % Dummy text

\blindtext % Dummy text

%------------------------------------------------

\section{Dispositivo experimental}

Maecenas sed ultricies felis. Sed imperdiet dictum arcu a egestas. 
\begin{itemize}
\item Donec dolor arcu, rutrum id molestie in, viverra sed diam
\item Curabitur feugiat
\item turpis sed auctor facilisis
\item arcu eros accumsan lorem, at posuere mi diam sit amet tortor
\item Fusce fermentum, mi sit amet euismod rutrum
\item sem lorem molestie diam, iaculis aliquet sapien tortor non nisi
\item Pellentesque bibendum pretium aliquet
\end{itemize}
\blindtext % Dummy text

Text requiring further explanation\footnote{Example footnote}.

%------------------------------------------------

\section{Resultados y an\'alisis}

\begin{table}
\caption{Example table}
\centering
\begin{tabular}{llr}
\toprule
\multicolumn{2}{c}{Name} \\
\cmidrule(r){1-2}
First name & Last Name & Grade \\
\midrule
John & Doe & $7.5$ \\
Richard & Miles & $2$ \\
\bottomrule
\end{tabular}
\end{table}

\blindtext % Dummy text

\begin{equation}
\label{eq:emc}
e = mc^2
\end{equation}

\blindtext % Dummy text

%------------------------------------------------

\section{Conclusiones}

\subsection{Subsection One}

A statement requiring citation \cite{Figueredo:2009dg}.
\blindtext % Dummy text

\subsection{Subsection Two}

\blindtext % Dummy text

%----------------------------------------------------------------------------------------
%	REFERENCE LIST
%----------------------------------------------------------------------------------------

\begin{thebibliography}{99} % Bibliography - this is intentionally simple in this template

\bibitem[Figueredo and Wolf, 2009]{Figueredo:2009dg}
Figueredo, A.~J. and Wolf, P. S.~A. (2009).
\newblock Assortative pairing and life history strategy - a cross-cultural
  study.
\newblock {\em Human Nature}, 20:317--330.
 
\end{thebibliography}

%----------------------------------------------------------------------------------------

\end{document}
